% !TeX root = ../thuthesis-example.tex

\chapter{总结与展望}
\section{本文总结}
本文首先介绍了 IoTDB 中目前写入数据的几种方式,以及优化多设备多记录写入机制性能的必要性与现实意义。随后介绍了 IoTDB 用于写入多设备多记录的写入方式—— InsertRecords 写入的完整执行流程。然后本文使用 IoT Benchmark 对 IoTDB 的 InsertRecords 和 InsertTablet 写入方式进行对比测试,并使用性能分析工具对执行两种不同写入方式时 IoTDB 的运行状况进行了分析,发现了 InsertRecords 写入方式的性能瓶颈。基于这些瓶颈,本文从客户端、RPC 层和存储引擎层优化了 InsertRecords 写入机制。最后,本文使用 IoT Benchmark 和用户场景的模拟负载对优化后的 InsertRecords 写入机制的性能进行了验证。本文的主要贡献如下:
\begin{enumerate}
  \item 本工作完整地分析了现有 IoTDB 多设备多记录写入机制的执行流程,并在压力测试场景下使用性能分析工具量化了各个环节的开销,确定了已有多设备多记录写入机制的瓶颈所在,然后从设计和实现的角度分析了导致瓶颈出现的原因。
  \item 本工作从客户端、RPC 层、存储引擎三个方面优化了 IoTDB 的多设备多记录写入机制,包括客户端请求格式转换、数据预处理、写入请求列式序列化、存储引擎批量化执行、写前日志压缩等,解决了原有多设备多记录写入机制的瓶颈,从而提高了写入性能。
  \item 本工作使用压力测试工具,对各个优化措施所带来的性能提升进行了测试,验证了优化措施的有效性。为了贴近用户的真实场景,本工作设计并实现了一套用户负载采样以及模拟运行程序,通过对若干用户的真实负载进行采样,生成了一系列模拟负载,然后使用这些模拟负载对优化前后的 IoTDB 的多设备多记录写入机制进行了测试。实验结果表明,本工作的优化让 IoTDB 的多设备多记录写入机制在性能提升的同时节约了系统资源。
\end{enumerate}
\section{本文不足与展望}
本文设计的新 InsertRecords 写入机制虽然解决了过去的一些性能瓶颈,但仍然存在可以优化的地方,包括:
\begin{enumerate}
  \item 在海量序列场景下,多设备多记录写入机制的瓶颈主要在于元数据校验,因为在这种场景下,元数据缓存的命中率较低,每次元数据校验都可能需要从远端拉取信息,导致写入性能下降。因此,如何在这种场景下优化元数据校验是未来的研究方向。
  \item 目前 IoTDB 每个 Data Region 的写入仍然是串行的,这是为了保证共识协议的一致性而设计的。目前工业界有一些工作提出了一些并行化的共识协议,如 PolarFS 所使用的 ParallelRaft\cite{cao2018polarfs}。未来 IoTDB 可以进一步研究如何在保证一致性的情况下解除对单个 Data Region 的写入串行化限制,提高写入性能。
  \item I/O 瓶颈仍然是在时序数据库写入场景下制约 IoTDB 写入性能的主要因素之一,如何通过优化 I/O 资源来提高系统的写入性能仍然需要大量的工作,例如更高效的 I/O 调度、异步 I/O、I/O 合并等。
\end{enumerate}