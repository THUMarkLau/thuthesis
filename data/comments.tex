% !TeX root = ../thuthesis-example.tex

\begin{comments}
% \begin{comments}[name = {指导小组评语}]
% \begin{comments}[name = {Comments from Thesis Supervisor}]
% \begin{comments}[name = {Comments from Thesis Supervision Committee}]
  刘旭鑫同学针对Apache IoTDB多设备批量行式写入接口优化开展研究工作,其选题具有重要的理论与应用价值。

主要工作包括:
\begin{enumerate}
  \item 完整分析了现有多设备批量行式写入机制的执行流程,在压力测试场景下使用性能分析工具量化了各个处理环节的性能开销;
  \item 设计并实现了协同客户端、 RPC 层、存储引擎等三个方面的优化写入机制;
  \item 设计并实现了一套用户负载采样以及模拟运行程序,以验证上面工作的有效性。
\end{enumerate}

\end{comments}
