% !TeX root = ../thuthesis-example.tex

\begin{acknowledgements}
岁月不居,时节如流,一转眼我的二十余年求学生涯即将画上句号。回首我的学生时代,十分有幸认识了很多良师益友,正是得益于他们的悉心指引与长久陪伴,我才能在浩瀚的学海中乘风破浪,找到前行的正确航向。在此,我衷心地向他们表达我最诚挚的谢意。

感谢导师王建民教授。王老师是工业大数据领域知名的专家,在他的带领下,大数据系统软件国家工程实验室发展蒸蒸日上,IoTDB 作为实验室的名片受到了学术界和工业界的广泛认可。在攻读研究生期间,我深受实验室中浓厚的工程和学术氛围感染,不仅显著提升了自身的工程技能,也让我树立起了在系统领域有所作为的决心。

感谢黄向东副研究员和乔嘉林博士。黄老师和乔博士是 IoTDB 社区的核心人物,他们不仅在工程和学术上帮助了我,也在生活中给予了我非常多的关心。他们的创业历程引起了我对商业和技术的思考,让我的技术观念更加成熟。

感谢 Apache IoTDB 社区。研究生三年我有幸参与社区的开发,并主导了一些重要的工作。这些经历不仅让我的工程技能得到了提升,也让我学会了与他人的协作。在开源社区的实践中,我深刻体会到了技术文档的重要性、代码评审的严谨性以及编码风格的规范性,这些都是教科书之外无法获取的宝贵经验。

衷心感谢我亲爱的家人,他们始终是我坚强的后盾。在我的学生时代,我从未因物质匮乏而分心,这都得益于家人对我毫无保留的支持与付出。我的爷爷是一位退休的中学校长,他一生桃李满天下,而我有幸成为他最关心、最引以为豪的学生。在我求学的道路上,遇到了许多恩师,但爷爷对我的影响无疑是最为深远和重要的。他的言传身教,将是我一生最宝贵的财富。

感谢我的女友刘思怡,异地七年的时间见证了我们真挚的感情,她的关心与鼓励也总是我努力和奋斗的动力。

感谢刘明辉、权思屹、周钰坤、周沛辰、朱海铭、喻思成同学,他们是与我同届的研究生,他们的存在不仅让我的研究生生涯变得有趣充实,与他们的交流也时常让我收获颇多。感谢刘雍同学,他是我在清华期间最好的朋友,他为人风趣幽默,在学术上充满真知,是难得的良师益友。感谢谭新宇、苏宇荣、田原、孙泽嵩、魏祥威等学长,他们与我亦师亦友,和他们的交流总能让我有所收获。感谢陈文科同学,他是我自高中以来的好友,在我需要帮助时他总不辞辛劳,希望他顺利毕业拿到医学博士学位。


\end{acknowledgements}
