\section{时序行式写入接口形式与度量分析\label{sec:chap1-sec2}}
\subsubsection{性能指标}
对写入接口的性能度量主要有两个指标:吞吐和延迟。

吞吐指的是单位时间内可以写入多少数据量,延迟指的是每个写入请求从发送到执行成功所经过的时间。延迟的单位通常为秒或者毫秒,在单个请求数据量发生变化时延迟也会发生变化,数据量越大延迟就会越高。所以,在衡量写入延迟性能的好坏时,只有固定单个写入请求的数据量才有对比的意义。

在 IoTDB 中,数据的基本单位是每个时间序列中一个时间戳和一个数据值组成的数值对,我们称这样的一个数值对为一个数据点。根据这样的定义,通过行式接口写入的一行记录中可能包含了若干个数据点。如果吞吐的测量单位使用行每秒(row per second),由于每一行的数据点数可能不同,系统实际写入的数据量可能有很大的差别。因此,对吞吐量测量较为合理的方法为使用写入数据点的总数量除以写入时间,衡量单位时间内写入的点数,单位为点每秒(points per second)。

\subsubsection{资源使用指标}
除了衡量性能的指标外,对写入性能的分析还可以从资源使用的指标来进行分析。数据在写入过程中主要会用到四个方面的资源:CPU、内存、网络、磁盘。

CPU 资源的使用情况可以通过利用率来进行衡量,这一指标通常为一个小于等于 $100\%$ 的数字,代表一段时间内 CPU 实际工作时间占总时间的百分比,可以提现 CPU 的繁忙程度。但是,在写入吞吐不同时,CPU 的利用率可能也会有所差别。因此,更为合适的指标是将单位时间内 CPU 的利用率除以单位时间内写入的数据点数,该结果体现的是写入单个数据点所需要耗费的 CPU 资源。该值越小,代表了写入机制对 CPU 的利用效率越高。

内存资源的使用情况可以用进程对内存的占用量来衡量。由于 IoTDB 使用 Java 编程语言构建,而 Java 的内存管理并不会主动释放无用的内存,而是等待合适的时机对内存进行垃圾回收(Garbage Collection,GC)\cite{manson2005java},我们无法实时监测 IoTDB 的内存使用量。鉴于此,我们可以使用 JVM 内存垃圾回收的次数以及垃圾回收使用时间占进程总运行时间的百分比来衡量系统内存资源的使用情况。一般而言,垃圾回收的次数越频繁、垃圾回收所使用的时间占进程运行总时间的比例越高,则说明系统的内存资源越紧张。

网络资源的使用情况可以使用吞吐(Throughput)或者每秒处理的数据包数(Packets Per Second,PPS)来衡量。吞吐指的是单位时间内有多少数据量通过网络被发送或者接收,PPS 则代表网卡在单位时间内处理了多少个网络数据包。这两个指标越高,则代表网络资源的使用情况越紧张。

磁盘资源的使用情况可以使用吞吐或者每秒钟进行读写的次数(Input/Output Operations Per Second,IOPS)来衡量。吞吐指的是每秒钟有多少数据量通过被写入到磁盘上,IOPS 则代表磁盘在每秒钟内进行了多少次读写操作。这两个指标越高,则代表磁盘资源的使用情况越紧张。

\subsubsection{其他指标}
Apache IoTDB 在执行数据写入的请求时,需要经过 RPC 反序列化、时间序列 ID 和路径校验、元数据校验、权限校验、MPP 框架处理、内存控制处理、监控框架记录性能指标、写预写日志(Write Ahead Log,WAL)、写入内存表、更新内存索引这一系列过程。在这些过程中,写入线程有可能会因为对共享变量的竞争或等待磁盘 IO 而被阻塞,在不被阻塞的时间内只有将数据写入内存表这个过程才是真正记录数据的步骤,其他步骤则是一些预处理和辅助步骤。当数据写入内存表这一步骤所耗费的时间占写入总时间的比例越大,说明辅助步骤的效率越高,系统的大部分资源被真实地利用在了数据的记录上。因此,我们可以设置一个名为\textbf{写入有效利用率}的指标,其含义为写入过程中真正记录数据所占用的时间除以处理整个写入所占用的时间。